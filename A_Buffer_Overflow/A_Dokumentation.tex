\documentclass[12pt,a4paper]{article}
\usepackage[utf8]{inputenc}
\usepackage{amsmath,amsthm, amssymb}
\usepackage{amsfonts}
\usepackage{amssymb}
\usepackage{graphicx} % For pictures
\usepackage[export]{adjustbox}
\usepackage{pdfpages} % for pdfs
\usepackage{listings}
\usepackage{xcolor}
\usepackage{caption}

\definecolor{mGreen}{rgb}{0,0.6,0}
\definecolor{mGray}{rgb}{0.5,0.5,0.5}
\definecolor{mPurple}{rgb}{0.58,0,0.82}
\definecolor{backgroundColour}{rgb}{0.95,0.95,0.95}

\lstdefinestyle{CStyle}{
    backgroundcolor=\color{backgroundColour},   
    commentstyle=\color{mGreen},
    keywordstyle=\color{magenta},
    numberstyle=\tiny\color{mGray},
    stringstyle=\color{mPurple},
    basicstyle=\footnotesize,
    breakatwhitespace=false,         
    breaklines=true,                 
    captionpos=b,                    
    keepspaces=true,                 
    numbers=left,                    
    numbersep=5pt,                  
    showspaces=false,                
    showstringspaces=false,
    showtabs=false,                  
    tabsize=2,
    language=C
}
\lstset{style=CStyle}
\usepackage{titling}
\usepackage[margin=0.75in]{geometry}
\setlength{\droptitle}{-7em}   % This is your set screw
\date{}
\setlength{\parindent}{0em}

\author{Lara Quitte, Tino Jeromin}
\title{Versuch A: Buffer Overflow}


\begin{document}
\maketitle


\section*{Hintergrundwissen}
Ein Programm in C ist in folgende Speicherbereiche aufgeteilt:
\begin{itemize}
\item Stack
\item Heap
\item Uninitialized data
\item Initialized data 
\item Text
\end{itemize}

Diese Bereiche sollen im Folgenden kurz näher betrachtet werden. Ausserdem soll ein Buffer-Overflow sowie Stack- und Heap-Overflow erläutert werden.
\bigskip

\textbf{Heap} \\
Der Heap wird in C genutzt, um dynamisch Speicher für Variablen, Arrays, etc zu reservieren.
Es wird also während der Laufzeit des Programms Speicher reserviert und wieder freigegeben.
Dies geschieht mittels 'malloc', 'realloc' und 'calloc'. 
Die so angelegten Variablen und ihr Speicherplatz existieren bis sie vom Programm mittels 'free' wieder
freigegeben werden. Werden die angelegten Variablen nicht wieder freigegeben, kann es zu Memory Leaks
kommen, d. h. das Programm reserviert immer mehr Speicher, gibt ihn aber nicht wieder frei, bis kein 
Speicherplatz mehr verfügbar ist.
\bigskip

\textbf{Stack} \\
Auf dem Stack werden die Funktionen und die dazugehörigen Informationen gespeichert.
Ruft also ein Programm eine Funktion auf, wird die Rücksprungadresse, die übergebenen Parameter und 
alle lokalen Variablen auf den Stack gelegt. Diese Informationen werden zusammen Stack Frame genannt, 
was jede Funktion besitzt.
Ist eine Funktion beendet, wird der Stack Frame wieder vom Stack genommen und das Programm springt 
zu der Funktion, die diese Funktion aufgerufen hat.
Das Legen einer Variable auf den Stack wird als 'push' bezeichnet und das entfernen vom Stack als 'pop'.
Das Betriebssystem speichert typischerweise mittels eines Zeigers den Anfang des Stacks und das aktuell 
oberste Element des Stacks.
Der Stack ist je nach Betriebssystem und CPU unterschiedlich implementiert.
\bigskip

\textbf{Uninitialized data}\\
Hier werden alle globalen und statischen Variablen gespeichert, welche nicht initialisiert wurden, oder 
zu 0 initialisiert wurden.
\bigskip

\textbf{Initialized data}\\
Hier werden alle globalen und statischen Variablen gespeichert, welche vom Programmierer initialisiert 
wurden. 
Zusätzlich gibt es einen Bereich für read-write Zugriff und einen für read-only Zugriff.
\bigskip

\textbf{Text Segment}\\
In disem Segment wird der Code des Programms gespeichert, also die ausführbaren Befehle.
Üblicherweise ist dieses Segment read-only, damit es nicht im Falle von stack- oder heap-overflows 
überschrieben wird.
\bigskip

\textbf{Buffer Overflow}\\
Ein Buffer in C ist typischerweise ein Array mit einer festen Größe. Wird nun eine Datenmenge, die 
größer ist als der Buffer, in den Buffer gespeichert, ohne dass der Speicherplatz überprüft wird, 
dann kommt es zu einem Buffer-Overflow. Dabei wird der Speicherplatz hinter dem Buffer auch noch
beschrieben, welcher gar nicht mehr für den Buffer reserviert ist und andere Informationen enthalten 
kann, die überschrieben werden.
\bigskip

\textbf{Stack Overflow}\\
Bei einem Stack-Overflow werden Daten an eine Stelle geschrieben, die nicht dafür vorgesehen sind. 
Zum Beipiel wenn in einen Buffer mehr Daten geschrieben werden sollen, als dafür reserviert ist.
Ein Angriff, welcher sich einen Stack-Overflow zu Nutze macht, basiert beispielsweise auf einem 
Buffer-Overflow. Dabei werden Daten in einen Buffer geschrieben, der auf dem Stack liegt.\\
Da bei einem Buffer-Overflow über den Buffer hinaus auf den Stack geschrieben wird, kann dadurch auch 
die Rücksprungadresse verändert werden. 
Verlangt ein Programm ein Array(String) als Parameter, wie zum Beispiel ein Programm, was über eine Shell
ausgeführt wird und kopiert dieses Array ohne die Länge zu überprüfen, dann wird das Eingabe-Array auf 
den Stack kopiert und überschreibt je nach Länge auch die Rücksprungadresse. Die Funktion 'strcpy' 
verwendet zum Beipiel kein Bounds-Checking.
Auf diese Weise kann die Rücksprungadresse so verändert werden, dass die in den Buffer zeigt, der ja 
das Eingabe-Array enthält. Wenn dieses Eingabe-Array nun aus ausführbaren Instruktionen besteht, werden 
diese statt der eigentlichen ausgeführt.\\
So kann beispielsweise eine Shell aufgerufen werden, auf welcher dann beliebige weitere Programme ausgeführt 
werden können. 
Wurde das ausgenutze Programm als root installiert, würde die aufgerufene Shell sogar root-Rechte haben.
\bigskip

\textbf{Heap Overflow}\\
Ein Heap-Overflow tritt auf, wenn mehr Daten in den Heap geschrieben werden sollen, als Speicherplatz für 
den Heap verfügbar ist.
\bigskip

\section*{Teil 1: Non-Control Data Attack (Buffer Overflow)}

Um den ersten Teil des Versuches durchzuführen, haben wir folgende Schritte durchgeführt:

\begin{enumerate}
\item Öffne Ghidra
\item Schliesse Help Window
\item Neues Project erstellen
\item Import 'ncd' executable
\item Öffne ncd im CodeBrowser
\item Symbol Tree $\rightarrow$ Exports $\rightarrow$ main
\item Window $\rightarrow$ Decompile: main 
\end{enumerate}

\begin{center}
\includegraphics[scale=0.3]{ghidra.png}
\captionof{figure}{ncd decompiled in Ghidra}
\end{center}

Nach Schritt 7 erhält man folgenden Code:\\

\begin{lstlisting}[style=CStyle]
ulong main(int iParm1,undefined8 *puParm2)

{
  int iVar1;
  size_t sVar2;
  char local_30 [20];
  int local_1c;
  undefined8 *local_18;
  int local_10;
  uint local_c;
  
  local_c = 0;
  if (iParm1 == 3) {
    local_1c = 0;
    local_18 = puParm2;
    local_10 = iParm1;
    sVar2 = strlen((char *)puParm2[1]);
    if (sVar2 == 8) {
      iVar1 = strncmp((char *)local_18[1],"password",8);
      if (iVar1 == 0) {
        local_1c = 1;
      }
    }
    strcpy(local_30,(char *)local_18[2]);
    if (local_1c == 0) {
      printf("Your password is not valid.\n");
    }
    else {
      printf("Hello %s. Your password is correct.\n",local_30);
    }
    local_c = 0;
  }
  else {
    local_18 = puParm2;
    local_10 = iParm1;
    printf("Usage: %s <password> <name>\n",*puParm2);
    local_c = 1;
  }
  return (ulong)local_c;
}
\end{lstlisting}
\bigskip

Wählt man als Parameter irgendeinen Namen und als Passwort eine Zeichenfolge, die länger als 20
Zeichen ist, hält das Programm die Kombination für korrekt.
\bigskip

Mithilfe des von Ghidra erzeugten C-Codes lässt sich die Funktionsweise des Programms gut nachvollziehen.
Zu Beginn der 'main' Funktion werden verschiedene lokale Variablen deklariert, für welche 
der benötigte Speicherplatz auf dem Stack hintereinander reserviert wird.
Dabei wird der Buffer 'local\textunderscore 30' mit einer Größe von 20 vor dem Integer 'local\textunderscore lc' 
auf den Stack gelegt.
Die kritische Stelle ist in Zeile 25, wenn mittels 'strcpy' unser eingegebenes Passwort 
in den Buffer 'local\textunderscore 30' kopiert werden soll. Da 'strcpy' jedoch kein Bounds-Checking 
betreibt, überschreiben Passworteingaben, die länger als 20 Zeichen sind, die Variablen, 
die nach dem Buffer auf dem Stack liegen. In diesem Fall als erstes 'local\textunderscore lc'. 
'local\textunderscore lc' wird aber in Zeile 26 in einer if-Abfrage dazu genutzt wird, die Korrektheit 
des Passwortes festzustellen. 
Ist 'local\textunderscore lc' gleich '0', ist das Passwort falsch. Sonst ist das Passwort richtig. 
Da durch den Buffer-Overflow 'local\textunderscore lc' mit einem Zeichen überschrieben wurde, 
welches nicht '0' ist, verhält sich das Programm, als ob das Passwort richtig wäre.
\bigskip

Eine Möglichkeit, den Angriff zu verhindern, ist, 'strncpy' statt 'strcpy' zu verwenden.
Alternativ könnte vor dem Aufruf von 'strcpy' die Länge des Strings überprüft werden.
Ausserdem könnten skalare Variablen vor Buffern deklariert werden, sodass die skalaren 
Variablen vor den Buffern auf dem Stack liegen und so bei einem Buffer-Overflow keine 
skalaren Variablen überschrieben werden können. 
Eine weitere drastische Maßnahme wäre, eine andere Programmiersprache wie Java oder eine 
interpretierte Sprache zu verwenden, da bei diesen Buffer-Overflows nicht möglich sind.
\bigskip

Eine Möglichkeit, Buffer-Overflows zu entdecken, ist jede Einagbemöglichkeit mit sehr 
langen Eingaben zu testen, um einen Buffer-Overflow zu provozieren.
Eine einfache Möglichkeit, jeden möglichen Buffer-Overflow zu entdecken, gibt es jedoch nicht. 
Um Programme auf diese Sicherheitslücke zu untersuchen, muss sich jede Verwendung von Buffern 
genau angeschaut werden. 
\bigskip

Um zu verhindern, dass ein Angreifer auf den gesamten Arbeitsspeicher zugreifen kann, benutzt 
Linux zum Beipiel Stack Canaries. Dies sind bestimmte Werte, die auf dem Stack zwischen 
der Rücksprungadresse und den lokalen Variablen gelegt werden. Diese Werte werden regelmäßig, 
zum Beispiel vor dem Sprung zur Rücksprungadresse, überprüft. Ist der Wert verändert worden, 
wird das Programm abgebrochen. SO wird verhindert, dass die Rücksprungadresse verändert wird. 
Zudem verwendet Linux Paging und Virtual Memory. Der Arbeitsspeicher wird also in Blöcke 
aufgeteilt, welche eine virtuelle Adresse haben. Versucht ein Programm auf eine Adresse 
ausserhalb des eigenen Blocks zuzugreifen, entsteht ein Page Fault, sodass das Programm 
unterbrochen wird. 
\bigskip


\end{document}