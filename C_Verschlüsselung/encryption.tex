
\documentclass[12pt,a4paper]{article}
\usepackage[utf8]{inputenc}
\usepackage{amsmath,amsthm, amssymb}
\usepackage{amsfonts}
\usepackage{amssymb}
\usepackage{graphicx} % For pictures
\usepackage[export]{adjustbox}
\usepackage{pdfpages} % for pdfs
\usepackage{listings}
\usepackage{xcolor}
\usepackage{caption}

\definecolor{mGreen}{rgb}{0,0.6,0}
\definecolor{mGray}{rgb}{0.5,0.5,0.5}
\definecolor{mPurple}{rgb}{0.58,0,0.82}
\definecolor{backgroundColour}{rgb}{0.95,0.95,0.95}

\lstdefinestyle{CStyle}{
    backgroundcolor=\color{backgroundColour},   
    commentstyle=\color{mGreen},
    keywordstyle=\color{magenta},
    numberstyle=\tiny\color{mGray},
    stringstyle=\color{mPurple},
    basicstyle=\footnotesize,
    breakatwhitespace=false,         
    breaklines=true,                 
    captionpos=b,                    
    keepspaces=true,                 
    numbers=left,                    
    numbersep=5pt,                  
    showspaces=false,                
    showstringspaces=false,
    showtabs=false,                  
    tabsize=2,
    language=C
}
\lstset{style=CStyle}
\usepackage{titling}
\usepackage[margin=0.75in]{geometry}
\setlength{\droptitle}{-7em}   % This is your set screw
\date{}
\setlength{\parindent}{0em}

\author{Lara Quitte, Tino Jeromin}
\title{Versuch C: Verschlüsselung}

\begin{document}
\maketitle
\section{GnuPG}
\subsection{Einwegfunktionen/Hashfunktionen}
\subsubsection{Notizen zur Durchführung}
\begin{itemize}
    \item \textbf{Welche Hashfunktionen werden unterstützt?}\\
    SHA1, RIPEMD160, SHA256, SHA384, SHA512, SHA224
    \item \textbf{Berechnen Sie einen Hashwert einer Datei mittels einer Hashfunktion Ihrer Wahl.}\\
    Vorgehen: 'gpg --print-mds dateiname' (gibt alle Hashwerte an, mit 'gpg --print-md SHA1 dateiname' kann man die Hashfunktion auswählen)
   \item \textbf{Erläutern Sie die von Ihnen ausgewählte Hashfunktion näher. Welche Unterschiede treten auf in Bezug auf andere Hashfunktionen?}\\
   Die als Secure Hash Standard (SHS) bezeichnete Norm spezifiziert den sicheren Hash-Algorithmus(SHA) mit einem Hash-Wert von 160 Bit Länge für beliebige digitale Daten von maximal 264-1Bit Länge. SH1 wird heutzutage nicht mehr als sicher bezeichnet, da nicht kollisionsresistent ist. Der Hashwert hat bei unterschiedlichen Funktionen eine unterschiedliche Länge.
   \item \textbf{Laden Sie eine Datei aus dem Internet, für die ein Hashwert zur Verfügung gestellt wird, und überprüfen Sie den mitgelieferten Hashwert. Was kann man aus der Gleichheit schließen? Und was kann nicht aus der Gleichheit schließen?}\\
   Wenn man von Kollisionsfreiheit ausgeht, dann ist...
   \begin{itemize}
       \item Hashwert unabhängig vom verwendeten Gerät und OS
       \item bei Verwendung der selben Hashfunktion der Hashwert bei der Datei immer gleich
   \end{itemize}
   \item \textbf{Welche der oben genannten Hashfunktionen sollte auf keinen Fall für den Zweck benutzt werden, sicherzustellen, dass die korrekte Datei ausgeliefert wurde?} MD5 sollte nicht genutzt werden, da dieses Verfahren veraltet ist und eine geringe Geschwindigkeit aufweist. Insbesondere hinsichtlich Kollisionsangriffen ist die Funktion nicht mehr ausreichend.
\end{itemize}



\subsubsection{Hashfunktionen}
Hashfunktionen bilden eindeutigen digitalen Fingerabdruck von Daten. Eine Hash-Funktion H ist eine Funktion die eine (variabel lange) Nachricht M auf einen Hash-Wert H(M) konstanter Länge abbildet.\\
Hash-Funktionen haben die folgenden Eigenschaften
\begin{itemize}
    \item Einweg-Funktion: H ist nicht invertierbar, d.h. bei gegebenem Hash-Wert H(M) kann M nicht ermittelt werden (d.h nicht mit realistischem Rechenaufwand leistbar)
    \item Kollisionsresistent: Es ist ebenfalls (praktisch) nicht möglich, ein paar verschiedener Nachrichten M und M’ zu finden, für die gilt : H(M) = H(M’)
\end{itemize}

\textbf{MD5:} \\
	Thirdly, similar to messages, you can also generate different files that hash to the same value so using MD5 as a file checksum is 'broken'.\\
\textbf{SHA512} \\
	Secure if used correctly, few attacks\\
\textbf{RIPEMD160} \\
Old, but unbroken. \\
Cryptographers generallly aim for 128 to 256 bits of security. RIPEMD only offers 160 / 2 = 80 bits due to the birthday bound, 
	  which approximately halves the amount of security offered. \\
\subsubsection{Kollision}
Eine Funktion (in diesem Zusammenhang fast immer eine Einwegfunktion) wird als kollisionsresistent bezeichnet, wenn es schwer ist, 
	  verschiedene Eingaben zu finden, die auf denselben Wert abgebildet werden. Insbesondere bei kryptographischen 
	  Hashfunktionen handelt es sich hierbei um eine übliche Anforderung, deren Bruch in der Regel als Bruch der 
	  kompletten Hashfunktion betrachtet wird. 
\subsubsection{Fragen}
\textbf{Welche wichtigen Eigenschaften haben Hashfunktionen?}
\begin{itemize}
    \item wenn Eingaben verschieden sind, ist auch der Hashwert unterschiedlich
    \item gleiche Eingaben führen immer zum selben Hashwert
    \item Einweg-Funktion
    \item Kollisionsresistent
    \item Schnelligkeit: Berechnung muss schnell erfolgen
\end{itemize}
\textbf{Wofür werden Hashfunktionen genutzt?}
\begin{itemize}
    \item eindeutigen Fingerabdruck von Daten bilden
    \item Prüfsummen erstellen
    \item Digitale Signaturen
    \item Speichern von Passwörtern
\end{itemize}
\bigskip
\subsection{symmetrische Verschlüsselung}
\subsubsection{Notizen zur Durchführung}
\begin{itemize}
    \item \textbf{Welche Algorithmen der symmetrischen Verschlüsselung beherrscht GnuPG?} IDEA, 3DES, CAST5, BLOWFISH, AES, AES192, AES256, TWOFISH, CAMELLIA128, CAMELLIA192, CAMELLIA256
    \begin{itemize}
        \item einen Text symmetrisch verschlüsseln und in eine Datei schreiben:  \textit{gpg -c -o output message}
        \item wieder entschlüsseln und in Standardeingabe schreiben: \textit{gpg -d output.txt}
        \item entschlüsselte Nachricht in eine Datei schreiben: \textit{gpg -o decrypt.txt -d output.txt}
        \item Vergleichen Sie die Größe der verschlüsselten und unverschlüsselten Datei. Was fällt auf und wie lässt es sich erklären? \textit{Die verschlüsselte Datei ist kleiner. Dies ist daruch zu erklären, dass die Daten zuerst automatisch komprimiert und dann verschlüsselt werden. }
    \end{itemize}
    
    \item \textbf{Verschlüsseln Sie eine Datei mit dem BLOWFISH-Algorithmus, wählen Sie ZIP als Kompressionalgorithmus und schreiben Sie die verschlüsselten Daten in eine Ausgabedatei.}
    BLOWFISH ist ein symmetrischer Blockverschlüsselungsalgorithmus mit einer Blocklänge von 64 Bit. Er garantiert eine fehlerfreie Umkehrung von Ver- und Entschlüsselung. Die Schlüssellänge kann zwischen 32 Bit und 448 Bit betragen. Aus diesen Schlüsselbits werden vor Beginn der Ver- oder Entschlüsselung die so genannten Rundenschlüssel P1 bis P18 und die Einträge in den S-Boxen erzeugt, insgesamt 4168 Byte.\\
    Generell zählen sowohl .zip als auch .rar zu den verlustfreien Kompressionsverfahren. Das heißt, dass die Datei nach dem Packen und anschließendem Entpacken zu 100 Prozent identisch mit dem Original wieder hergestellt wird. \\

    Anweisung: \textit{gpg -e --cipher-algo BLOWFISH --compress-algo ZIP -o compressed decrypt}
    
    \item \textbf{Verschlüsseln Sie eine Datei mit dem IDEA-Algorithmus, wählen Sie BZIP2 als Kompressionalgorithnmus und schreiben Sie die verschlüsselten Daten in eine Ausgabedatei.}
    IDEA benutzt eine Serie von acht identischen Transformationen, welche je einer Runde entsprechen, und einer Ausgabetransformation, welche einer halben Runde entspricht. Der Entschlüsselungsprozess entspricht dem Verschlüsselungsprozess in umgekehrter Form. Bei der Verschlüsselung wird der Klartext in 64 Bit große Blöcke unterteilt und der Schlüssel in Teilstücke zu je 16 Bit zerlegt. \\
    Bzip2 ist ein eigener Kompressionsalgorithmus. Er benutzt eine Kombination von verschiedenen Methoden, um die Datei zu komprimieren. Zuerst werden die Daten mit der Burrows-Wheeler-Transformation und der Move-To-Front-Transformation transformiert, wobei die Dateigröße sich noch nicht verändert. Danach wird die Datei mit der Run-Length- Kodierung und der Huffman-Kodierung komprimiert.\\
    
    Anweisung: \textit{gpg -e --cipher-algo IDEA --compress-algo BZIP2 -o compressed decrypt} \\
    Diese Datei ist etwas größer als die mit ZIP.
    
\end{itemize}

\subsubsection{symmetrischer Schlüssel}
\begin{itemize}
    \item Beide Parteien haben den gleichen Schlüssel
	\item Wird zum Ver- und Entschlüsseln benutzt
	\item Je länger der Schlüssel, desto sicherer
	\item Schneller als asymmetrisch
	\item Für jeden Kommunikationspartner ein Schlüssel - sehr viele Schlüssel
	\item symmetrische Verschlüsselung gewährleistet:
	\begin{itemize}
	    \item Geheimhaltung – weil eine verschlüsselte Nachricht nicht ohne Kenntnis des geheimen Schlüssels dechiffriert werden kann
	    \item Integrität – weil eine verschlüsselte Nachricht nicht ohne Kenntnis des geheimen Schlüssels so modifiziert werden kann, dass die Dechiffrierung normalen Klartext ergibt
	    \item Authentizität – weil eine verschlüsselte Nachricht, deren Dechiffrierung normalen Klartext liefert, nur von jemandem, der den geheimen Schlüssel kennt, stammen kann
	\end{itemize}
	\item mögliches Problem: Integrität und Authentizität sind eventuell bedroht durch Wiedereinspielen (replay) zuvor abgehörter (verschlüsselter) Nachrichten oder Nachrichtenteile
	\item mögliche Lösung: Nachrichten durchnumerieren Rückkoppelung ( gemäß Cipher Block Chaining, Output Feedback, Cipher Feedback)Nachricht mit Zeitangabe versehen
	\item verbleibende Schwäche: Geheimer Schlüssel ist Gruppenschlüssel von 2 (oder mehr) Partnern
	\item Verbindlichkeit ist daher nicht gewährleistet: A: „Ich habe von B diese Nachricht erhalten“
ist weder beweisbar noch widerlegbar.
\end{itemize}
\subsubsection{symmetrische Ver- und Entschlüsselung von Dateien/Texten}
\begin{itemize}
    \item der selbe Schlüssel wird zur Ver- und Entschlüsselung genutzt
    \item will man einen symmetrisch verschlüsselten Text weitergeben, so muss man auch den Schlüssel mitgeben (öffentlicher Schlüssel)
    \item das eigentliche Problem der symmetrischen Verschlüsselung ist die Weitergabe des Schlüssels, da ein sicherer Kanal für die Übermittlung nötig ist - E-Mail fällt also z.B weg
    \item da die Weitergabe also nicht immer trivial, daher asymmetrische Verschlüsselung oder Hybrid sicherer
    \item symmetrische Verschlüsselung sinnvoll für lokal genutzte Dateien, da Schlüssel dann nicht weitergegeben werden muss
    \item man benötigt: Datei, Schlüssel, Algorithmus
\end{itemize}
% ------------------------------------------------------------------------------------
% ------------------------------------------------------------------------------------
% ------------------------------------------------------------------------------------
\bigskip
\subsection{Asymmetrische/hybride Verschlüsselung}
\subsubsection{Notizen zur Durchführung}
\begin{itemize}
    \item Welche Algorithmen der asymmetrischen Verschlüsselung beherrscht GnuPG? RSA, ELG, DSA, ECDH, ECDSA, EDDSA
    \item Erzeugen Sie dann pro Gruppenmitglied mindestens ein Schlüsselpaar von sinnvoller Art und Größe. Erklären Sie, warum die von Ihnen gewählte Schlüsselgröße sinnvoll ist (die Defaulteinstellung ist KEIN Argument). Erzeugen Sie außerdem ein Widerrufszertifikat.
    \begin{itemize}
        \item gpg --gen-key Erstellt einen Schlüssel 
        \item gpg --output revoke.asc --gen-revoke 2183480FE2402B1A Erstellt ein Widerrufszertifikat zum jeweiligen Schlüssel
        \item Die Schlüssellänge ist 3072 Bits, dies entspricht etwa der AES und ist damit als sicher einzustufen.
    \end{itemize}
    \item Exportieren Sie dann jeweils Ihren öffentlichen Schlüssel im ASCII-Format und importieren den Ihres Gruppenmitglieds (an dieser Stelle sollten Sie sich klar machen, dass dieser Schlüsseltausch eine kritische Angelegenheit ist: Es sollte stets sichergestellt werden, dass der erhaltene Schlüssel tatsächlich zur richtigen Person gehört).
    \begin{itemize}
        \item Exportieren: gpg --export -a [UID]
        \item Importieren: gpg --import [Datei]
    \end{itemize}
    \item Verschlüsseln Sie jeweils eine Datei für Ihr Gruppenmitglied (welchen Schlüssel nehmen Sie?). Machen Sie sich klar, was in welcher Reihenfolge auf welche Art verschlüsselt wird (dies soll sich in der eingangs erwähnten Skizze wiederfinden). Entschlüsseln Sie diese wieder.
    \begin{itemize}
        \item gpg --encrypt -o encryptMessage --sign -r davidjer@hotmail.de -u 2183480FE2402B1A message\\ Datei wird mit private key verschlüsselt
        \item gpg encryptMessage\\ verschlüsselte Datei wird Empfänger entschlüsselt
    \end{itemize}
    \item Schicken Sie sich gegenseitig verschlüsselte E-Mails zu und lesen die entschlüsselte Nachricht des anderen. Wählen Sie dazu einen geeigneten E-Mail-Client und richten ihn entsprechend ein. Machen Sie sich klar, was in welcher Reihenfolge auf welche Art verschlüsselt wird.
    \begin{itemize}
        \item gpg --export-secret-keys --armor stu225209@mail.uni-kiel.de\textgreater mein-key.asc\\
        Exportieren der secret keys um sie in thunderbird zu importieren
    \end{itemize}

\end{itemize}
\subsubsection{Stichwörter}
\begin{itemize}
    \item asymmetrische Verschlüsselung
    \begin{itemize}
        \item Ein privater und ein öffentlicher Schlüssel für jeden Kommunikationspartner 
	    \item Eine mit dem public key verschlüsselte Nachricht kann nur mit dem private key entschlüsselt werden und andersherum
	    \item  asymmetrische Verschlüsselung ermöglicht durch die Verschlüsselung einer Nachricht mit dem öffentlichen Schlüssel die Vertraulichkeit bzw. Geheimhaltung der Nachricht, da der verschlüsselte Text nur mit dem privaten Schlüssel gelesen werden kann
	    \item Verschlüsselung mit dem öffentlichen Schlüssel garantiert jedoch nicht die Authentizität des Ursprungs der Nachricht, da der öffentliche Schlüssel von allen benutzt werden kann und allein an der Verschlüsselung nicht gesehen werden kann, von wem die Nachricht stammt
    \end{itemize}
    \item Schlüsselpaar
    \begin{itemize}
        \item Absender und Empfänger verwenden verschiedene Teile eines Schlüsselpaares
        \item jeder Nutzer erzeugt sein eigenes Schlüsselpaar bestehend aus einem öffentlichen und einem privaten Schlüssel
        \item der öffentliche Schlüssel ermöglicht es jedem, Daten für den Besitzer des privaten Schlüssels zu verschlüsseln, dessen digitale Signaturen zu prüfen oder ihn zu authentifizieren
        \item der private Schlüssel ermöglicht es seinem Besitzer, mit dem öffentlichen Schlüssel verschlüsselte Daten zu entschlüsseln, digitale Signaturen zu erzeugen oder sich zu authentisieren
    \end{itemize}
    \item Schlüsselring
    \begin{itemize}
        \item dient zur sicheren Speicherung und Verwaltung von Passwörtern
        \item Passwörter werden verschlüsselt gespeichert und mit einem Masterpasswort geschützt
        \item Auslesen der Passwörter ist auch beispielsweise nach dem Diebstahl eines Notebooks nicht möglich
        \item das Masterpasswort muss eine hohe Qualität haben und für niemanden zugänglich sein
    \end{itemize}
    \item Schlüsselimport
    \item Schlüsselexport
    \item öffentlicher Schlüssel: ermöglicht es jedem, Daten für den Besitzer des privaten Schlüssels zu verschlüsseln, dessen digitale Signaturen zu prüfen oder ihn zu authentifizieren
    \item privater Schlüssel: ermöglicht es seinem Besitzer, mit dem öffentlichen Schlüssel verschlüsselte Daten zu entschlüsseln, digitale Signaturen zu erzeugen oder sich zu authentisieren
    \item Schlüsselserver: bietet Zugang zu öffentlichen Schlüsseln, die in asymmetrischen Kryptosystemen dazu benutzt werden, einer Person verschlüsselte Mitteilungen – beispielsweise per E-Mail – zu senden oder ihre Signaturen zu verifizieren
    \item Bestandteile eines Schlüssels
    \item asymmetrische Ver- und Entschlüsselung von Dateien/Nachrichtens
\end{itemize}
\subsubsection{Fragen}
\textbf{Erklären Sie anhand Ihres Versuches zur verschlüsselten E-Mail, wie die hybride Verschlüsselung genau funktioniert.}\\
Pretty Good Privacy nutzt eine Variante des Public-Key-Systems, bei dem jeder Anwender einen öffentlichen Schlüssel besitzt, der beliebig weitergegeben werden kann. Dazu kommt ein privater Schlüssel, der nur dem Anwender bekannt ist. Um eine verschlüsselte Nachricht an jemand anderes zu versenden, benötigen Sie seinen öffentlichen Schlüssel. Wenn der Empfänger die Nachricht erhalten hat, kann er sie mit seinem privaten Schlüssel decodieren. Weil es sehr zeitaufwändig sein kann, eine komplette Nachricht zu verschlüsseln, setzt PGP auf einen schnelleren Algorithmus. Dieser verwendet einen zusätzlichen kürzeren Key, um die Nachricht zu verschlüsseln. Nur dieser kürzere Schlüssel wird dann mit dem öffentlichen Schlüssel des Empfängers verschlüsselt. Sowohl die chiffrierte Nachricht als auch der verschlüsselte kürzere Schlüssel werden dann an den Empfänger gesendet. Mit seinem privaten Schlüssel kann dieser zunächst den verschlüsselten Key entschlüsseln, und diesen dann dazu verwenden, die eigentliche Nachricht wieder lesbar zu machen.\\
Unter Hybrider Verschlüsselung, auch Hybridverschlüsselung genannt, versteht man eine Kombination aus asymmetrischer Verschlüsselung und symmetrischer Verschlüsselung. Dabei wählt der Sender einen zufälligen symmetrischen Schlüssel, der Session-Key bzw. Sitzungsschlüssel genannt wird. Mit diesem Session-Key werden die zu schützenden Daten symmetrisch verschlüsselt. Anschließend wird der Session-Key asymmetrisch mit dem öffentlichen Schlüssel des Empfängers verschlüsselt. Dieses Vorgehen löst das Schlüsselverteilungsproblem und erhält dabei den Geschwindigkeitsvorteil der symmetrischen Verschlüsselung.\\
\textbf{Nach welchen Kriterien haben Sie Ihre Verschlüsselungsalgorithmen ausgesucht und die Schlüssellänge bestimmt?}
\bigskip
\subsection{Signaturen}
Eine digitale Signatur, auch digitales Signaturverfahren, ist ein asymmetrisches Kryptosystem, 
	  bei dem ein Sender mit Hilfe eines geheimen Signaturschlüssels (dem Private Key) zu einer digitalen Nachricht 
	  (d. h. zu beliebigen Daten) einen Wert berechnet, der ebenfalls digitale Signatur genannt wird. Dieser Wert 
	  ermöglicht es jedem, mit Hilfe des öffentlichen Verifikationsschlüssels (dem Public Key) die nichtabstreitbare Urheberschaft und Integrität der Nachricht zu prüfen.
	   Die digitale Signatur sichert die Echtheit des Senders, da kein anderer den geheimen Schlüssel zum Signieren besitzt. Die digitale Signatur sichert jedoch nicht die Vertraulichkeit der Nachricht, da jeder den öffentlichen Schlüssel des Signierenden bekommen kann, um damit die Nachricht zu lesen. Daher kann man eine signierte Nachricht noch zusätzlich mit dem öffentlichen Schlüssel des Empfängers verschlüsseln, um sowohl Vertraulichkeit als auch Authentizität zu erreichen.
\subsubsection{Notizen zur Durchführung}
Wie in der Einleitung erwähnt, ist die Echtheit eines öffentlichen Schlüssels die Achillesferse des Systems. Deshalb gibt es die Möglichkeit, Schlüssel zu unterschreiben. Damit bestätigt der Unterzeichnende, daß der in der User ID angegeben User tatsächlich der Besitzer des Schlüssels ist.

Nachdem man mit gpg --edit-key UID den zu unterzeichnenden Schlüssel ausgewählt hat, kann man ihn mit dem Kommando sign unterschreiben.

Unterschreiben Sie nur Schlüssel von deren Echtheit sie sich überzeugt haben. Das kann geschehen, in dem man entweder den Schlüssel persönlich bekommen hat (zum Beispiel auf einer Keysigning Party), oder man über Telefon den Fingerprint vergleicht. Man sollte keinen Schlüssel nur deshalb unterschreiben, weil man den anderen Unterschriften vertraut.

Anhand der Unterschriften und des "ownertrusts" ermittelt GnuPG die Gültigkeit des Schlüssels. Der Ownertrust ist ein Wert mit dem der Benutzer festlegt, in welchem Maße er dem Schlüsselinhaber zutraut, andere Schlüssel verläßlich zu unterzeichnen. Die möglichen Abstufungen sind "gar nicht", "weiß nicht", "teilweise" und "vollständig". Wenn der Benutzer also einem anderem nicht traut, kann er GnuPG über diesen Mechanismus anweisen, dessen Unterschrift zu ignorieren. Der Ownertrust wird nicht im Schlüsselbund gespeichert, sondern in einer separaten Datei.
\begin{itemize}
    \item Schicken Sie jeweils dem anderen eine signierte Nachricht, die ansonsten unverschlüsselt ist und prüfen Sie die Ihnen zugesandte signierte Nachricht. Was passiert intern in welcher Reihenfolge? Macht es einen Unterschied, ob Sie --sign oder --clearsign benutzen?
    \begin{itemize}
        \item gpg -s -o signed message (\textbf{Sign}) Hier wird die eigentliche Nachricht so verändert, dass sie nicht mehr in einem Editor geöffnet werden kann. Die Nachricht ist noch lesbar. Das Dokument wird komprimiert und der Output ist ein Binary.
        \item  gpg --clearsign -o clearsigned message (\textbf{ClearSign}) Hier wird Signatur an den Text angehängt. Die Nachricht bleibt erhalten. Die Signatur wird als ASCII Zeichen geschrieben.
        \item wird die Nachricht zusätzlich vorher mit \textit{gpg -s -e -r stu225209@mail.uni-kiel.de -o signedEncrypted message} verschlüsselt und dann signiert, so erhält man eine verschlüsselte Datei, die außerdem signiert ist
        \item das Verschlüsseln einer clearsigned message ist nicht möglich, da nicht sinnvoll
        \item Importieren öffentlicher Schlüssel\\
gpg - -import Schlüssel (Nimmt öffentlichen Schlüssel Schlüssel ins Schlüsselbund auf) - mit save in gpg bestätigen
        \item Signieren des öffentlichen Schlüssels\\
        \textit{gpg - -edit-key UID} - danach  sign und save
        \item Mit \textit{gpg --list-sig davidjer@hotmail.de} kann man prüfen, wer dem angegebenen Schlüssel vertraut
        \item wird bei einer signierte Mail der Text verändert und ohne weitere Änderungen an der Signatur verschickt, so ist die Signatur am Ende nicht mehr vorhanden
        \item Verschlüsseln Sie eine Datei mit AES256 und signieren Sie diese (dieses sollte in einem einzigen Befehl stattfinden\\
        \textit{gpg -c -o aesSigned --cipher-algo AES256 -s message}
        
    \end{itemize}
    
\end{itemize}
\subsubsection{Stichwörter}
\begin{itemize}
    \item Signieren von Dateien/Nachrichten: Sicherstellung, dass eine bestimmte Datei/Nachricht von einem bestimmten Absender stammt (Authentizität) und nicht manipuliert wurde (Integrität) - Signaturen werden mit privaten Schlüsseln erstellt und es kann jede Art von Datei signiert werden
    \item Bestandteile eines Schlüssels: mehrere Eigenschaften wie Name und E-Mail des Eigentümers, den Schlüsseltyp, das Ablaufdatum usw.
    \item Fingerprint eines Schlüssels: eindeutige Folge von Buchstaben und Zahlen, mit welcher der Schlüssel identifiziert werden kann, wenn Eigenschaften mehrerer Schlüssel gleich sind 
    \item Signieren von Schlüsseln: man muss sich von der Echtheit des verwendeten Schlüssels zu überzeugen - dies geschieht mittels des Fingerprints oder einer Signatur
\end{itemize}
\subsubsection{Fragen}
\textbf{Welches Ziel verfolgt man bei der Signierung eines Schlüssels? Wie wird dieses Ziel technisch realisiert?} Die digitale bzw. elektronische Signatur ist eine schlüsselabhängige Prüfsumme, die von einer Nachricht oder einem Dokument in Kombination mit einem Schlüssel erzeugt wird. Wird die Signatur an eine Nachricht oder ein Dokument angehängt, dann gilt das als unterschrieben. Für digitale Nachrichten und Dokumente werden digitale Signaturen verwendet, um ihre Echtheit glaubhaft und prüfbar zu machen. Die Echtheit der Signatur kann elektronisch geprüft werden.\\
Um Nachrichten und Dokumente zu signieren muss dessen Ersteller die Nachricht mit seinem privaten Schlüssel "entschlüsseln". Der dabei entstandene "Klartext" ist die digitale Signatur. Sie wird an das Dokument angehängt. Die Signatur kann von jedem anderen mit dem öffentlichen Schlüssel des Erstellers "verschlüsselt" werden. Dabei entsteht die ursprüngliche Nachricht, die mit der unverschlüsselten Nachricht verglichen werden kann. Sind beide gleich, ist das Dokument unverändert und korrekt signiert.\\
\textbf{RSA als Signaturverfahren:}
Ein Teilnehmer "unterschreibt" eine Nachricht m, indem er sie mit seinem privaten Schlüssel d kodiert. Heraus kommt die Signatur s.
Er verschickt die Nachricht m zusammen mit der Signatur s.
Die Echtheit der Nachricht m und die Identität der Person kann durch die Signatur s und den öffentlichen Schlüssel von jedem überprüft werden.\\

Wenn ein Benutzer ein Dokument elektronisch unterschreibt, wird unter Nutzung des privaten Schlüssels des Unterzeichners eine Signatur erzeugt. Der private Schlüssel wird vom Unterzeichner geheim gehalten. Der mathematische Algorithmus arbeitet wie eine Chiffre und erzeugt Daten zu dem betreffenden Dokument, Hash genannt, und verschlüsselt die Daten. Die resultierenden verschlüsselten Daten sind die digitale Signatur. Die Signatur wird zudem mit einem Zeitstempel versehen. Wenn das Dokument nach der Unterzeichnung verändert wird, ist es ungültig.
\bigskip
\section{Festplattenverschlüsselung}
\subsection{verschlüsselte Container}
Container sind insbesondere geeignet, auf einer ansonsten nicht verschlüsselten Partition einen privaten verschlüsselten Bereich für sensible Daten anzulegen. Innerhalb eines Containers verwaltet TrueCrypt ein Dateisystem. Zum Lesen und Schreiben mountet TrueCrypt diese Datei. Unter Windows wird dazu ein neues virtuelles Laufwerk erstellt. Unter macOS und Linux wird der Container in ein beliebiges Verzeichnis eingehängt. Zugriffe auf das Laufwerk/das Verzeichnis unterscheiden sich nicht von Zugriffen auf andere, nicht durch TrueCrypt erzeugte Pendants. Die Ver- und Entschlüsselung übernimmt der TrueCrypt-Treiber im Hintergrund (englisch on the fly). Container können, wenn sie nicht eingebunden sind, wie normale Dateien behandelt werden, beispielsweise auf eine DVD gebrannt werden.
\subsubsection{Fragen}
\textbf{Betrachten Sie den Inhalt des Containers (also der Datei in ihrem Dateisystem und nicht der gemounteten virtuellen Festplatte) in einem Editor oder Kommandozeilenwerkzeig (cat, less, ...). Können Sie Informationen über die Dateien und ihre Inhalte finden?}


% ------------------------------------------------------------------------------------
% ------------------------------------------------------------------------------------
% ------------------------------------------------------------------------------------


\bigskip
\section{SSH Schlüssel}
\subsection{Stichwörter}
\begin{itemize}
    \item SSH Schlüssel
    \begin{itemize}
        \item Sicherer als Passwörter
	    \item Kein Passwort zum login nötig
	    \item Passwort wird nicht über das Netzwerk geschickt
	    \item Usernames and passwords are easily remembered, and if web login is possible, 
	  browsers can auto-fill these fields, making it even easier to log in.
	    \item Sichere Passwörter sind schwer zu merken und man muss sie regelmäßig ändern
    \end{itemize}
    \item Agenten: Bei jedem SSH-Login mit Public-Key-Authentifizierung muss die Passphrase des jeweiligen privaten Schlüssels eingegeben werden. Wenn solche Verbindungen häufiger aufgebaut werden müssen, macht es Sinn einen SSH Key-Agent zu verwenden. Dieser hält alle privaten Schlüssel dekodiert im Speicher, sobald man einmal die entsprechende Passphrase eingegeben hat. Ein ständiges Eintippen der Passphrase für jede neue SSH-Verbindung entfällt somit.
\end{itemize}
\subsection{Fragen}
\textbf{Was für Vorteile und Nachteile hat ein SSH Schlüssel gegenüber einem Passwort-basierten Login?}
\bigskip

% ------------------------------------------------------------------------------------
% ------------------------------------------------------------------------------------
% ------------------------------------------------------------------------------------



\section{SSH Tunnel, Port Forwarding, SOCKS Proxy}
\subsection{Stichwörter}
\begin{itemize}
    \item SSH Tunnel \\
        Tunnel meint nun, dass Daten von einem dritten Rechner, zum Beispiel eine Webseite, vom SSH-Server angefordert 
	  und dann über die SSH-Verbindung an Ihren lokalen Rechner weitergeleitet werden. Angenommen, Ihr Netzwerkadministrator 
	  hat die Webseite "example.com" gesperrt. Sie können aber auf Ihren SSH-Server "ssh.meinsshserver.de" zugreifen - und 
	  dieser wiederum auch auf "example.com". Dann sagen Sie Ihrem SSH-Server, dass er "example.com" an Ihren lokalen 
	  Rechner leiten soll - und Sie können sie dann im Browser ansehen. Das Beste: SSH-Verbindungen sind wie gesagt 
	  verschlüsselt, sprich es ist für niemanden ersichtlich, welche Daten, hier also die Webseite von "example.com",
	  Sie sich anschauen. \\
	   A Secure Shell (SSH) tunnel consists of an encrypted tunnel created through an SSH protocol connection. 
	  Users may set up SSH tunnels to transfer unencrypted traffic over a network through an encrypted channel.
    \item Port Forwarding
    \begin{itemize}
        \item Local port forwarding: connections from the SSH client are forwarded via the SSH server, then to a destination server
	    \item Remote port forwarding: connections from the SSH server are forwarded via the SSH client, then to a destination server
	    \item Dynamic port forwarding: connections from various programs are forwarded via the SSH client, then via the SSH server, 
	  and finally to several destination servers
    \end{itemize}
    \item SOCKS Proxies: Das SOCKS-Protokoll ist ein Internet-Protokoll.
Dieses erlaubt Anwendungen, Daten protokollunabhänging über einen Proxyserver zu leiten.
SOCKS-Proxies sind dadurch universell einsetzbar z.B. zum Abrufen von Webseiten, versenden von Mails,
selbst die Daten von online Spielen lassen sich so mit speziellen Programmen (siehe weiter unten) über Proxyserver leiten.
\end{itemize}
\subsection{Fragen}
\textbf{Was sind die Vorteile und Nachteile von SSH Tunnel gegenüber einem VPN Tunnel?}
\begin{itemize}
    \item VPN benötigt Client-software und einen VPN-Server
	\item SSH Tunnel ist einfacher einzurichten
	\item Anfänger können problemlos eine Verbindung zu einem VPN herstellen, das Einrichten eines VPN-Servers ist jedoch komplexer
	\item SSH-Tunnel ist der einfachste Weg, den Netzwerkverkehr zu verschlüsseln und zu tunneln
	\item bei beiden ist die Verschlüsselung gleich gut 
\end{itemize}
\section{Quellen}
\begin{itemize}
    \item https://www.dr-datenschutz.de/hashwerte-und-hashfunktionen-einfach-erklaert/
    \item http://www.inf.fu-berlin.de/lehre/SS04/SySi/folien/KryptografieTeil2.pdf
    \item https://de.wikipedia.org/wiki/Verschlüsselung
    \item https://help.gnome.org/users/seahorse/stable/misc-key-fingerprint.html.de
    \item https://graphentech.io/ge/vpn-vs-ssh-tunnel-was-ist-sicherer/
    \item https://www.elektronik-kompendium.de/sites/net/1910131.htm
\end{itemize}
\end{document}
